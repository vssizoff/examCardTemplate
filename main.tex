\documentclass{article}
\usepackage[a5paper, total={5.5in, 7.6in}]{geometry}
\usepackage{pgfpages}
\pgfpagesuselayout{2 on 1}
\usepackage{enumitem}
% Для иллюстраций
\usepackage{asymptote}
\usepackage{tikz}
\usetikzlibrary{positioning,graphs,calc,decorations.pathmorphing,shapes,arrows.meta,arrows,shapes.misc,fit,matrix}
% Вставка картинок
\usepackage{graphicx}
\usepackage[russian]{babel}
\usepackage{amssymb, amsmath, amsthm, mathtools}
\usepackage{fontspec}
\setmainfont{CMU Serif}
\setsansfont{CMU Sans Serif}
\setmonofont{CMU Typewriter Text}

\newcounter{problem}[section]
\newenvironment{problem}[0]{
    \bigskip
    \refstepcounter{problem}
    \BIG\par \theproblem)
}

\begin{document}

    [image]
    
    \thispagestyle{empty}

    \newpage

\begin{titlepage}
    \vspace*{30pt}

    \begin{center}
        {\Huge Зачёт №1.Четырёхугольники} \bigskip

        \textsl{\large Для: \_\_\_\_\_\_\_\_} \bigskip

        \textsl{\large Составил: \_\_\_\_\_\_\_\_}
    \end{center}

    \vspace*{\fill}

    {\LARGE Теория: \underline{\hspace{2cm}}}

    \vspace*{50px}

    {\LARGE Практика: \underline{\hspace{2cm}}}

    \vspace*{50px}

    {\LARGE Лекционка: \underline{\hspace{2cm}}}

    \vspace*{30pt}
\end{titlepage}

\thispagestyle{empty}

\newpage

    {\LARGE Теория}

    \begin{large}
        \begin{enumerate}[label=\arabic*)]
            \item Многоугольник (определение).
            Стороны, периметр, диагональ многоугольника.
            \item Число диагоналей $n$-угольника.
            \item Выпуклые и невыпуклые многоугольники.
            Правильный многоугольник (определения).
            \item Теорема о сумме внутренних углов выпуклого многоугольника.
            \item Теорема о сумме внешних углов выпуклого многоугольника,
            взятых по одному при каждой вершине.
            \item Дельтоид (определение).
            Свойство диагоналей дельтоида.
            \item Параллелограмм (определение).
            Свойства параллелограмма (свойство противоположных сторон, углов, диагоналей).
            \item Свойства биссектрисы (биссектрис) угла параллелограмма.
            \item Угол между высотами параллелограмма.
            \item Четыре признака параллелограмма.
            \item Свойство и признак прямоугольного треугольника.
            \item Прямоугольник (определение, свойства, признаки).
            \item Ромб (определение, свойства, признаки). Квадрат.
            \item Средняя линия треугольника (определение, свойства, признаки).
            \item Теорема Вариньона.
            \item Трапеция (определение), виды трапеции.
            \item Средняя линия трапеции (определение, свойства).
            \item Признаки средней линии трапеции.
            \item Равнобедренная трапеция (определение, свойства, признаки).
            \item Теорема Фалеса.
            \item Задача о делении отрезка на $n$ равных частей.
        \end{enumerate}
    \end{large}

\newpage

    {\LARGE Задачи}

    \begin{problem}
        Задача 1. Задача 1. Задача 1. Задача 1. Задача 1. Задача 1. Задача 1. Задача 1. Задача 1. Задача 1. Задача 1. Задача 1. Задача 1. Задача 1. Задача 1. Задача 1. Задача 1. Задача 1. Задача 1. Задача 1. Задача 1. Задача 1. Задача 1. Задача 1. Задача 1. Задача 1. Задача 1. Задача 1. Задача 1. Задача 1. Задача 1. Задача 1. Задача 1. Задача 1. Задача 1. Задача 1. Задача 1. Задача 1. Задача 1. Задача 1. Задача 1.
    \end{problem}

    \begin{problem}
        Задача 2. Задача 2. Задача 2. Задача 2. Задача 2. Задача 2. Задача 2. Задача 2. Задача 2. Задача 2. Задача 2. Задача 2. Задача 2. Задача 2. Задача 2. Задача 2. Задача 2. Задача 2. Задача 2. Задача 2. Задача 2. Задача 2. Задача 2. Задача 2. Задача 2. Задача 2. Задача 2. Задача 2. Задача 2. Задача 2. Задача 2. Задача 2. Задача 2. Задача 2. Задача 2. Задача 2. Задача 2. Задача 2. Задача 2. Задача 2. Задача 2. Задача 2. Задача 2. Задача 2. Задача 2. Задача 2. Задача 2. Задача 2.
    \end{problem}

    \begin{problem}
        Задача 3. Задача 3. Задача 3. Задача 3. Задача 3. Задача 3. Задача 3. Задача 3. Задача 3. Задача 3. Задача 3. Задача 3. Задача 3. Задача 3. Задача 3. Задача 3. Задача 3. Задача 3. Задача 3. Задача 3. Задача 3. Задача 3. Задача 3. Задача 3. Задача 3. Задача 3. Задача 3. Задача 3. Задача 3. Задача 3. Задача 3. Задача 3. Задача 3. Задача 3. Задача 3. Задача 3. Задача 3. Задача 3. Задача 3. Задача 3. Задача 3. Задача 3. Задача 3.
    \end{problem}

    \begin{problem}
        Задача 4. Задача 4. Задача 4. Задача 4. Задача 4. Задача 4. Задача 4. Задача 4. Задача 4. Задача 4. Задача 4. Задача 4. Задача 4. Задача 4. Задача 4. Задача 4. Задача 4. Задача 4. Задача 4. Задача 4. Задача 4. Задача 4. Задача 4. Задача 4. Задача 4. Задача 4. Задача 4. Задача 4. Задача 4. Задача 4. Задача 4. Задача 4. Задача 4. Задача 4. Задача 4. Задача 4. Задача 4. Задача 4. Задача 4. Задача 4. Задача 4. Задача 4. Задача 4. Задача 4. Задача 4. Задача 4. Задача 4.
    \end{problem}

    \begin{problem}
        Задача 5. Задача 5. Задача 5. Задача 5. Задача 5. Задача 5. Задача 5. Задача 5. Задача 5. Задача 5. Задача 5. Задача 5. Задача 5. Задача 5. Задача 5. Задача 5. Задача 5. Задача 5. Задача 5. Задача 5. Задача 5. Задача 5. Задача 5. Задача 5. Задача 5. Задача 5. Задача 5. Задача 5. Задача 5. Задача 5. Задача 5. Задача 5. Задача 5. Задача 5. Задача 5. Задача 5. Задача 5. Задача 5. Задача 5. Задача 5. Задача 5. Задача 5. Задача 5. Задача 5. Задача 5.
    \end{problem}

    \begin{problem}
        Задача 6. Задача 6. Задача 6. Задача 6. Задача 6. Задача 6. Задача 6. Задача 6. Задача 6. Задача 6. Задача 6. Задача 6. Задача 6. Задача 6. Задача 6. Задача 6. Задача 6. Задача 6. Задача 6. Задача 6. Задача 6. Задача 6. Задача 6. Задача 6. Задача 6. Задача 6. Задача 6. Задача 6. Задача 6. Задача 6. Задача 6. Задача 6. Задача 6. Задача 6. Задача 6. Задача 6. Задача 6. Задача 6. Задача 6. Задача 6. Задача 6. Задача 6. Задача 6. Задача 6. Задача 6. Задача 6.
    \end{problem}

\thispagestyle{empty}

\end{document}