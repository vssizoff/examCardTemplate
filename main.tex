\documentclass{article}
\usepackage[a4paper, total={7.6in, 11in}]{geometry}
\usepackage{pgfpages}
\pgfpagesuselayout{2 on 1}
\usepackage{enumitem}
% Для иллюстраций
\usepackage{asymptote}
\usepackage{tikz}
\usetikzlibrary{positioning,graphs,calc,decorations.pathmorphing,shapes,arrows.meta,arrows,shapes.misc,fit,matrix,intersections,through,backgrounds,angles,quotes,decorations.markings}
% Вставка картинок
\usepackage{graphicx}
\usepackage[russian]{babel}
\usepackage{amssymb, amsmath, amsthm, mathtools}
\usepackage{fontspec}
\usepackage{textcomp}
\usepackage{wrapfig}
\setmainfont{CMU Serif}
\setsansfont{CMU Sans Serif}
\setmonofont{CMU Typewriter Text}

\begin{document}

[image]

    \thispagestyle{empty}

    \newpage

    \begin{titlepage}
        \vspace*{30pt}

        \begin{center}
            {\scalebox{3}{Зачёт №2. Площади многоугольников.}\\ \scalebox{3}{Теорема Пифагора.}} \bigskip

        \textsl{\huge Для: <имя младшего>} \bigskip

        \textsl{\huge Составил: <имя старшего>}
        \end{center}

        \vspace*{\fill}

        \scalebox{2.5}{Теория: \underline{\hspace{1cm}}}

        \vspace*{80px}

        \scalebox{2.5}{Практика: \underline{\hspace{1cm}}}

        \vspace*{80px}

        \scalebox{2.5}{Лекционка: \underline{\hspace{1cm}}}

        \vspace*{80pt}
    \end{titlepage}

    \thispagestyle{empty}

    \newpage

    \scalebox{3}{Теория}

    {\fontsize{16}{20}\selectfont
\begin{enumerate}[label=\arabic*)]
    \item Определение площади многоугольника.
    \item Равновеликие и равносоставленные фигуры. Связь.
    \item Площадь квадрата.
    \item Площадь прямоугольника.
    \item Площадь параллелограмма.
    \item Площадь треугольника. Площадь прямоугольного треугольника.
    \item Отношение площадей треугольников:
    \begin{enumerate}[label=\arabic*)]
        \item с равными высотами
        \item с равными основаниями
        \item с равными углами
    \end{enumerate}
    \item Площадь трапеции (2 формулы).
    \item Площадь четырехугольника, диагонали которого взаимно перпендикулярны. Площадь ромба.
    \item ОПОРНЫЕ ЗАДАЧИ (8 задач).
    \item Среднее квадратичное в трапеции.
    \item Обобщение теоремы Вариньона.
    \item Теорема Пифагора (2 доказательства).
    \item Теорема, обратная теореме Пифагора. Следствие из теоремы Пифагора.
    \item Площадь равностороннего треугольника.
    \item Свойство и признак четырёхугольника, диагонали которого взаимно перпендикулярны.
    \item Формула Герона.
    \item Нахождение площади трапеции:
    \begin{enumerate}[label=\arabic*)]
        \item по четырем сторонам
        \item по основаниям и диагоналям
    \end{enumerate}
    \item Определения синуса, косинуса, тангенса и котангенса острого угла прямоугольного треугольника. Корректность определения.
    \item Основные тригонометрические тождества. Формулы приведения.
    \item Табличные значения тригонометрических функций для углов 30, 45, 60.
\end{enumerate}}

    \newpage

    \scalebox{3}{Задачи}

    {\newcounter{problem}[section]
\newenvironment{problem}[0]{
    \bigskip
    \refstepcounter{problem}
    \fontsize{21}{27}\selectfont \par \textnumero \theproblem.
}

\begin{problem}
Задача 1. Задача 1. Задача 1. Задача 1. Задача 1. Задача 1. Задача 1. Задача 1. Задача 1. Задача 1. Задача 1. Задача 1.
\end{problem}

\begin{problem}
Задача 2. Задача 2. Задача 2. Задача 2. Задача 2. Задача 2. Задача 2. Задача 2. Задача 2. Задача 2. Задача 2. Задача 2.
\end{problem}

\begin{problem}
Задача 3. Задача 3. Задача 3. Задача 3. Задача 3. Задача 3. Задача 3. Задача 3. Задача 3. Задача 3. Задача 3. Задача 3.
\end{problem}

\begin{problem}
Задача 4. Задача 4. Задача 4. Задача 4. Задача 4. Задача 4. Задача 4. Задача 4. Задача 4. Задача 4. Задача 4. Задача 4.
\end{problem}

\begin{problem}
Задача 5. Задача 5. Задача 5. Задача 5. Задача 5. Задача 5. Задача 5. Задача 5. Задача 5. Задача 5. Задача 5. Задача 5.
\end{problem}

\begin{problem}
Задача 6. Задача 6. Задача 6. Задача 6. Задача 6. Задача 6. Задача 6. Задача 6. Задача 6. Задача 6. Задача 6. Задача 6.
\end{problem}}

    \thispagestyle{empty}

\end{document}