\fontsize{16}{20}\selectfont
\begin{enumerate}[label=\arabic*)]
    \item Многоугольник (определение).
    Стороны, периметр, диагональ многоугольника.
    \item Число диагоналей $n$-угольника.
    \item Выпуклые и невыпуклые многоугольники.
    Правильный многоугольник (определения).
    \item Теорема о сумме внутренних углов выпуклого многоугольника.
    \item Теорема о сумме внешних углов выпуклого многоугольника,
    взятых по одному при каждой вершине.
    \item Дельтоид (определение).
    Свойство диагоналей дельтоида.
    \item Биссектриса угла и серединный перпендикуляр как ГМТ плоскости.
    \item Параллелограмм (определение).
    Свойства параллелограмма (свойство противоположных сторон, углов, диагоналей).
    \item Свойства биссектрисы (биссектрис) угла параллелограмма.
    \item Угол между высотами параллелограмма.
    \item Четыре признака параллелограмма.
    \item Опорная задача про параллелограмм, вписанный в другой параллелограмм
    \item Свойство и признак прямоугольного треугольника.
    \item Прямоугольник (определение, свойства, признаки).
    \item Ромб (определение, свойства, признаки). Квадрат.
    \item Средняя линия треугольника (определение, свойства, признаки).
    \item Теорема Вариньона.
    \item Трапеция (определение), виды трапеции.
    \item Средняя линия трапеции (определение, свойства).
    \item Признаки средней линии трапеции.
    \item Равнобедренная трапеция (определение, свойства, признаки).
    \item Теорема Фалеса.
    \item Задача о делении отрезка на $n$ равных частей.
\end{enumerate}