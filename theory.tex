\fontsize{16}{20}\selectfont
\begin{enumerate}[label=\arabic*)]
    \item Определение площади многоугольника.
    \item Равновеликие и равносоставленные фигуры. Связь.
    \item Площадь квадрата.
    \item Площадь прямоугольника.
    \item Площадь параллелограмма.
    \item Площадь треугольника. Площадь прямоугольного треугольника.
    \item Отношение площадей треугольников:
    \begin{enumerate}[label=\arabic*)]
        \item с равными высотами
        \item с равными основаниями
        \item с равными углами
    \end{enumerate}
    \item Площадь трапеции (2 формулы).
    \item Площадь четырехугольника, диагонали которого взаимно перпендикулярны. Площадь ромба.
    \item ОПОРНЫЕ ЗАДАЧИ (8 задач).
    \item Среднее квадратичное в трапеции.
    \item Обобщение теоремы Вариньона.
    \item Теорема Пифагора (2 доказательства).
    \item Теорема, обратная теореме Пифагора. Следствие из теоремы Пифагора.
    \item Площадь равностороннего треугольника.
    \item Свойство и признак четырёхугольника, диагонали которого взаимно перпендикулярны.
    \item Формула Герона.
    \item Нахождение площади трапеции:
    \begin{enumerate}[label=\arabic*)]
        \item по четырем сторонам
        \item по основаниям и диагоналям
    \end{enumerate}
    \item Определения синуса, косинуса, тангенса и котангенса острого угла прямоугольного треугольника. Корректность определения.
    \item Основные тригонометрические тождества. Формулы приведения.
    \item Табличные значения тригонометрических функций для углов 30, 45, 60.
\end{enumerate}